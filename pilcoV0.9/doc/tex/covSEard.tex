
% This LaTeX was auto-generated from an M-file by MATLAB.
% To make changes, update the M-file and republish this document.



    
    

\subsection*{covSEard.m} 

\begin{par}
Squared Exponential covariance function with Automatic Relevance Detemination (ARD) distance measure. The covariance function is parameterized as:
\end{par} \vspace{1em}
\begin{par}
k(x\^{}p,x\^{}q) = sf2 * exp(-(x\^{}p - x\^{}q)'*inv(P)*(x\^{}p - x\^{}q)/2)
\end{par} \vspace{1em}
\begin{par}
where the P matrix is diagonal with ARD parameters ell\_1\^{}2,...,ell\_D\^{}2, where D is the dimension of the input space and sf2 is the signal variance. The hyperparameters are:
\end{par} \vspace{1em}
\begin{par}
loghyper = [ log(ell\_1)              log(ell\_2)               .              log(ell\_D)              log(sqrt(sf2)) ]
\end{par} \vspace{1em}
\begin{par}
For more help on design of covariance functions, try "help covFunctions".
\end{par} \vspace{1em}
\begin{par}
(C) Copyright 2006 by Carl Edward Rasmussen (2006-03-24)
\end{par} \vspace{1em}

\begin{lstlisting}
function [A, B] = covSEard(loghyper, x, z)
\end{lstlisting}


\subsection*{Code} 


\begin{lstlisting}
if nargin == 0, A = '(D+1)'; return; end          % report number of parameters

persistent K;

[n D] = size(x);
ell = exp(loghyper(1:D));                         % characteristic length scale
sf2 = exp(2*loghyper(D+1));                                   % signal variance

if nargin == 2
  K = sf2*exp(-sq_dist(diag(1./ell)*x')/2);
  A = K;
elseif nargout == 2                              % compute test set covariances
  A = sf2*ones(size(z,1),1);
  B = sf2*exp(-sq_dist(diag(1./ell)*x',diag(1./ell)*z')/2);
else                                                % compute derivative matrix

  % check for correct dimension of the previously calculated kernel matrix
  if any(size(K)~=n)
    K = sf2*exp(-sq_dist(diag(1./ell)*x')/2);
  end

  if z <= D                                           % length scale parameters
    A = K.*sq_dist(x(:,z)'/ell(z));
  else                                                    % magnitude parameter
    A = 2*K;
    clear K;
  end
end
\end{lstlisting}
