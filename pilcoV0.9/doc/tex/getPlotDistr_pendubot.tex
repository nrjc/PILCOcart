
% This LaTeX was auto-generated from an M-file by MATLAB.
% To make changes, update the M-file and republish this document.



    
    
      \subsection{getPlotDistr\_pendubot.m}

\begin{par}
\textbf{Summary:} Compute means and covariances of the Cartesian coordinates of the tips both the inner and outer pendulum assuming that the joint state $x$ of the cart-double-pendulum system is Gaussian, i.e., $x\sim N(m, s)$
\end{par} \vspace{1em}

\begin{verbatim}   function [M1, S1, M2, S2] = getPlotDistr_pendubot(m, s, ell1, ell2)\end{verbatim}
    \begin{par}
\textbf{Input arguments:}
\end{par} \vspace{1em}
\begin{verbatim}m       mean of full state                                    [6 x 1]
s       covariance of full state                              [6 x 6]
ell1    length of inner pendulum
ell2    length of outer pendulum\end{verbatim}
\begin{verbatim}Note: this code assumes that the following order of the state:
       1: pend1 angular velocity,
       2: pend2 angular velocity,
       3: pend1 angle,
       4: pend2 angle\end{verbatim}
\begin{par}
\textbf{Output arguments:}
\end{par} \vspace{1em}
\begin{verbatim}M1      mean of tip of inner pendulum                         [2 x 1]
S1      covariance of tip of inner pendulum                   [2 x 2]
M2      mean of tip of outer pendulum                         [2 x 1]
S2      covariance of tip of outer pendulum                   [2 x 2]\end{verbatim}
\begin{par}
Copyright (C) 2008-2013 by Marc Deisenroth, Andrew McHutchon, Joe Hall, and Carl Edward Rasmussen.
\end{par} \vspace{1em}
\begin{par}
Last modification: 2013-03-27
\end{par} \vspace{1em}


\subsection*{High-Level Steps} 

\begin{enumerate}
\setlength{\itemsep}{-1ex}
   \item Augment input distribution to complex angle representation
   \item Compute means of tips of pendulums (in Cartesian coordinates)
   \item Compute covariances of tips of pendulums (in Cartesian coordinates)
\end{enumerate}

\begin{lstlisting}
function [M1, S1, M2, S2] = getPlotDistr_pendubot(m, s, ell1, ell2)
\end{lstlisting}


\subsection*{Code} 


\begin{lstlisting}
% 1. Augment input distribution
[m1 s1 c1] = gTrig(m, s, [3 4], [ell1, ell2]); % map input through sin/cos
m1 = [m; m1];        % mean of joint
c1 = s*c1;           % cross-covariance between input and prediction
s1 = [s c1; c1' s1]; % covariance of joint

% 2. Compute means of tips of pendulums (in Cartesian coordinates)
M1 = [-m1(5); m1(6)];                 % [-l*sin(t1), l*cos(t1)]
M2 = [-m1(5) + m1(7); m1(6) + m1(8)]; % [-l*(sin(t1)-sin(t2)),l*(cos(t1)+cos(t2))]

% 2. Put covariance matrices together (Cart. coord.)
% first set of coordinates (tip of 1st pendulum)
s11 = s1(5,5);
s12 = -s1(5,6);
s22 = s1(6,6);
S1 = [s11 s12; s12' s22];

% second set of coordinates (tip of 2nd pendulum)
s11 = s1(5,5) + s1(7,7) - s1(5,7) - s1(7,5);    % ell1*sin(t1) + ell2*sin(t2)
s22 = s1(6,6) + s1(8,8) + s1(6,8) + s1(8,6);    % ell1*cos(t1) + ell2*cos(t2)
s12 = -(s1(5,6) + s1(5,8) + s1(7,6) + s1(7,8));
S2 = [s11 s12; s12' s22];

% make sure we have proper covariances (sometimes numerical problems occur)
try
  chol(S1);
catch
  warning('matrix S1 not pos.def. (getPlotDistr)');
  S1 = S1 + (1e-6 - min(eig(S1)))*eye(2);
end

try
  chol(S2);
catch
  warning('matrix S2 not pos.def. (getPlotDistr)');
  S2 = S2 + (1e-6 - min(eig(S2)))*eye(2);
end
\end{lstlisting}
