%set the plot display orientation
%synatax: \tdplotsetdisplay{\theta_d}{\phi_d}
\tdplotsetmaincoords{65}{100}

%define polar coordinates for some vector
%TODO: look into using 3d spherical coordinate system
\pgfmathsetmacro{\magW}{4}
\pgfmathsetmacro{\psiW}{-30}
\pgfmathsetmacro{\theW}{10}
\pgfmathsetmacro{\phiW}{15}

%start tikz picture, and use the tdplot_main_coords style to implement the display 
%coordinate transformation provided by 3dplot
\footnotesize
\begin{tikzpicture}[scale=0.85,tdplot_main_coords]

%set up some coordinates ... GET FROM MATLAB
%-----------------------
\coordinate (O) at (0,0,0);

\coordinate (B) at (3,5.1962,0);
\coordinate (W) at (3.2256,5.0659,1.4772);
\coordinate (Wt) at (3.6767,4.8054,4.4316);
\coordinate (Wtp) at (3.2256,5.0659,4.4316);
\coordinate (Wbp) at (3.2256,5.0659,0);
\coordinate (T) at (3.2969,3.6799,5.7578);
\coordinate (F) at (3.2612,4.3729,3.6175);

%draw figure contents
%--------------------


% draw wheel
\tdplotsetrotatedcoords{\psiW}{100}{0} % add 90 to j angle
\tdplotsetrotatedcoordsorigin{(W)}
\draw[tdplot_rotated_coords,very thick,black!30,fill=black!30,fill opacity=0.2] (0,0,0) circle (1.5);
\draw[tdplot_rotated_coords,fill=black!30,black!30] (0,0,0) circle (0.12);

% draw frame
\draw[very thick,black!30] (W) -- (T);
\tdplotsetrotatedcoordsorigin{(F)}
\tdplotsetrotatedcoords{\psiW}{100}{0}
\draw[tdplot_rotated_coords,black!30,fill=black!30] (0,0,0) circle (0.12);

% draw turntable
\tdplotsetrotatedcoords{\psiW}{\theW}{\phiW}
\tdplotsetrotatedcoordsorigin{(T)}
\draw[tdplot_rotated_coords,very thick,fill=black,fill opacity=0.2] (0,0,0) circle (1.5);
\draw[tdplot_rotated_coords,fill=black] (0,0,0) circle (0.12);

% draw forces and moments
\tdplotdrawarcarrow[red,thick,dashed]{(T)}{0.5}{50}{360}{}{}
\tdplotsetrotatedcoords{0}{0}{0}
\tdplotsetrotatedcoordsorigin{(T)}
\draw[tdplot_rotated_coords,thick,-latex,red,dashed] (0,0.8,0.8) -- (0,0.05,0.05);
\draw[tdplot_rotated_coords,thick,-latex,red] (0,0,-0.06) -- (0,0,-1.2);

\node at (0,2.9,2.9) {\color{red}$W_\te{t}$};
\node at (0,4,4.7) {\color{red}$G_\te{t}$};
\node at (0,2.3,4) {\color{red}$P_\te{t}$};

\end{tikzpicture}
