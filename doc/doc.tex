\documentclass{article}

\usepackage{geometry}
\usepackage{dsfont}
\usepackage{mathtools}
\usepackage{amssymb}
\usepackage{amsmath}
\usepackage{color}
\usepackage{xcolor}
\usepackage{textcomp}

\usepackage{listings}

\definecolor{listinggray}{gray}{0.9}
\definecolor{lbcolor}{rgb}{0.9,0.9,0.9}
\lstset{
	%backgroundcolor=\color{lbcolor},
	tabsize=2,
	rulecolor=,
	language=matlab,
        basicstyle=\scriptsize,
        upquote=true,
        aboveskip={1.5\baselineskip},
        columns=fixed,
        showstringspaces=false,
        extendedchars=true,
        breaklines=true,
        prebreak = \raisebox{0ex}[0ex][0ex]{\ensuremath{\hookleftarrow}},
        %frame=single,
        showtabs=false,
        showspaces=false,
        showstringspaces=false,
        identifierstyle=\ttfamily,
        keywordstyle=\color[rgb]{0,0,1},
        commentstyle=\color[rgb]{0.133,0.545,0.133},
        stringstyle=\color[rgb]{0.627,0.126,0.941},
}

%\newcommand{\tt}[1]{\texttt{#1}}
\renewcommand{\vec}[1]{\boldsymbol{#1}}  % vector
\newcommand{\mat}[1]{\boldsymbol{#1}}    % matrix
\newcommand{\inv}{^{-1}}
\newcommand{\T}{^{\top}}
\newcommand{\E}{\mathds{E}}
\newcommand{\var}{\mathds{V}}
\newcommand{\cov}{\mathrm{cov}}
\newcommand{\R}{\mathds{R}}
\newcommand{\polpar}{\theta}
\newcommand{\path}{\texttt{<path>/}}



\title{Code Documentation} 

\author{Marc Peter Deisenroth, Andrew McHutchon, Joe Hall, Carl Edward
  Rasmussen}

\date{\today}

\begin{document}

\maketitle

\section{Flow}

\begin{enumerate}
\item \texttt{applyController}
\begin{enumerate}
\item determine start state
\item generate rollout
\begin{enumerate}
\item compute control signal $\pi(\vec x_t)$
\item simulate dynamics (or apply control to real robot)
\item transition to state $\vec x_{t+1}$
\end{enumerate}
\end{enumerate}
\item \texttt{trainDynModel}
\item \texttt{learnPolicy}
\begin{enumerate}
\item call gradient-based non-convex optimizer; minimize
  \texttt{value} with respect to policy parameters
\begin{enumerate}
\item \texttt{propagated:} compute successor state distribution
  $p(\vec x_{t+1})$ and gradients $\partial p(\vec
  x_{t+1})/\partial\vec\theta$ with respect to the policy parameters
\begin{enumerate}
\item trigonometric augmentation of the state distribution $p(\vec
  x_t)$
\item compute distribution of preliminary (unsquashed) policy
  $p(\tilde\pi(\vec x_t))$
\item compute distribution of squashed (limited-amplitude) policy
  $p(\pi(\vec x_t))=p(\vec u_t)$
\item determine successor state distribution $p(\vec x_{t+1})$ using
  GP prediction (\texttt{gp*})
\end{enumerate}
\item \texttt{loss:} compute expected cost $\E_{\vec x}[c(\vec x)]$
  and its partial derivatives $\partial/\partial p(\vec x)$
\end{enumerate}
\end{enumerate}
\end{enumerate}


\section{Software Package Overview}

This software package implements the \textsc{pilco} algorithm. The
package contains the following directories

\begin{itemize}
\item \texttt{base:} Root directory. Contains basic files and all
  other directories.
\item \texttt{control:} Directory that implements several controllers.
\item \texttt{doc:} Documention
\item \texttt{gp:} Everything that has to do with Gaussian processes
  (training, predictions, sparse GPs etc.)
\item \texttt{loss:} Several loss functions
\item \texttt{scenarios:} Different scenarios. Each scenario is
  packaged in a separate directory with all scenario-specific files
\item \texttt{util:} Utility files
\end{itemize}

\section{\texttt{Base} Directory}


\section{\texttt{Control} Directory}
The control directory is located at \path\texttt{control}. The
controllers compute the (unconstrained) control signals
$\tilde\pi(\vec x))$.

All controllers expect the following inputs
\begin{enumerate}
\item \texttt{policy:} A struct with the following fields
\begin{itemize}
\item \texttt{fcn}
\item \texttt{p:} policy parameters
\end{itemize}
\item \texttt{m:} $\E[\vec x]\in\R^D$ The mean of the state
  distribution $p(\vec x)$
\item \texttt{s:} $\var[\vec x]\in\R^{D\times D}$ The covariance
  matrix of the state distribution $p(\vec x)$
\end{enumerate}



The controller functions compute
\begin{enumerate}
\item \texttt{M:} $\E[\tilde\pi(\vec x)]\in\R^F$ The mean of the
  predicted (unconstrained) control signal
\item \texttt{S:} $\var[\tilde\pi(\vec x)]\in\R^{F\times F}$ The
  covariance matrix of the predicted (unconstrained) control signal
\item \texttt{V:} $\var(\vec x)\inv\cov[\vec x,\tilde\pi(\vec x)]$ The
  cross-covariance between the (input) state $\vec x$ and the control
  signal $\tilde\pi(\vec x)$, pre-multiplied with $\var(\vec x)\inv$,
  the inverse of the covariance matrix of $p(\vec x)$.
\item Derivatives. The derivatives of all output arguments with
  respect to all input arguments are computed:
\begin{itemize}
\item \tt{dMdm:} $\partial M/\partial m\in\R^{F\times D}$ The
  derivative of the mean of the predicted control with respect to the
  mean of the state distribution.
\item \tt{dSdm:} $\partial S/\partial m\in\R^{F^2\times D}$ The
  derivative of the covariance of the predicted control with respect to the
  mean of the state distribution.
\item \tt{dVdm:} $\partial V/\partial m\in\R^{DF\times D}$ The
  derivative of the cross-covariance $V$ with respect to the
  mean of the state distribution.
\item \tt{dMds:} $\partial M/\partial s\in\R^{F\times D^2}$ The
  derivative of the mean of the predicted control with respect to the
  covariance of the state distribution.
\item \tt{dSds:} $\partial S/\partial m\in\R^{F^2\times D^2}$ The
  derivative of the covariance of the predicted control with respect to the
  covariance of the state distribution.
\item \tt{dVds:} $\partial V/\partial m\in\R^{DF\times D^2}$ The
  derivative of the cross-covariance $V$ with respect to the
  covariance of the state distribution.
\item \tt{dMdp:} $\partial M/\partial\polpar\in\R^{F\times |\vec\polpar|}$ The
  derivative of the mean of the predicted control with respect to the
  policy parameters $\vec\polpar$.
\item \tt{dSdp:} $\partial S/\partial\polpar\in\R^{F^2\times |\vec\polpar|}$ The
  derivative of the covariance of the predicted control with respect to the
  policy parameters $\vec\polpar$.
\item \tt{dVdp:} $\partial V/\partial\polpar\in\R^{DF\times |\vec\polpar|}$ The
  derivative of the cross-covariance $V$ with respect to the
  policy parameters $\vec\polpar$.
\end{itemize}
\end{enumerate}

\section{\texttt{GP} Directory}


\section{\texttt{Loss} Directory}

\section{\texttt{Scenarios} Directory}
\lstinputlisting{../scenarios/cartPole/settings.m}

\section{\texttt{Util} Directory}


\section{value}
\lstinputlisting{../base/value.m}


\section{propagate}
\lstinputlisting{../base/propagate.m}

\section{GP Predictions at Uncertain Inputs: gp0}
\lstinputlisting{../gp/gp0.m}


\section{Linear Controller: conlin}
\lstinputlisting{../control/conlin.m}

\end{document}
%%% Local Variables: 
%%% mode: latex
%%% TeX-master: t
%%% End: 
