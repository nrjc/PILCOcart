\documentclass{article}
\renewcommand{\rmdefault}{psbx}
\usepackage[utf8]{inputenc}
\usepackage[T1]{fontenc}
\usepackage{textcomp}
\usepackage{eulervm}
\usepackage{amsmath}
\usepackage{amssymb}

\setlength{\textwidth}{160mm}
\setlength{\oddsidemargin}{0mm}
\setlength{\parindent}{0 mm}

\newcommand{\bfa}{{\bf a}}
\newcommand{\bfb}{{\bf b}}
\newcommand{\bfm}{{\bf m}}
\newcommand{\bfs}{{\bf s}}
\newcommand{\bfz}{{\bf z}}
\newcommand{\bfx}{{\bf x}}
\newcommand{\E}{{\mathbb E}}
\newcommand{\V}{{\mathbb V}}

\title{Augment Gaussian with Trigonometric Functions}
\author{Carl Edward Rasmussen}
\date{October 20th 2008}

\begin{document}

\maketitle

In several contexts it is useful to be able to augment a joint
Gaussian distribution with the trigonometric functions sine and cosine
of one or more of its coordinates. The resulting distribution is not
Gaussian, but we can compute exactly the first and second (central)
moments of the augmented distribution. Additionally, the derivative of
these moments wrt.~the parameters of the joint distribution are also
computed.

Let $x$ be a $D$ dimensional Gaussian
\[
\bfx\;\sim\;{\cal N}(\bfa,\,A),
\]
which we want to augment by sine and cosine of $x_i, \forall i\in I$,
where $d$ is the number of elements in $I$, resulting in the
$D\!+\!2d$ dimensional joint Gaussian
\[
\left[\!\!\begin{array}{c}\bfx\\ \bfz\end{array}\!\!\right]\;\sim\;
{\cal N}\left(\!\left[\!\!\begin{array}{c}\bfa\\
      \bfb\end{array}\!\!\right],\,
\left[\!\!\begin{array}{cc}A&B\\B^\top&C\end{array}\!\!\right]\!\right).
\]
Below we derive expressions for the elements $\bfb, B$ and $C$.

For the means, we have for $i=1,\ldots,d$
\[
\begin{split}
b_{2i-1}\;&=\;\E[\sin(x_{I(i)})]\;=\;\exp(-A_{I(i),I(i)}/2)\sin(a_{I(i)}),\\
b_{2i}\;&=\;\E[\cos(x_{I(i)})]\;=\;\exp(-A_{I(i)I(i)}/2)\cos(a_{I(i)}).
\end{split}
\]
For the covariances we have for $i=1,\ldots,d,j=1,\ldots,D$
\[
\begin{split}
B_{j,2i-1}\;&=\;\exp(-\tfrac{1}{2}A_{I(i),I(i)})\cos(a_{I(i)})A_{j,I(i)},\\
B_{j,2i}\;&=\;-\exp(-\tfrac{1}{2}A_{I(i),I(i)})\sin(a_{I(i)})A_{j,I(i)},
\end{split}
\]
and for $i,j=1,\ldots,d$, $i\neq j$
\[
\begin{split}
C_{2i-1,2i-1}\;&=\;q_i(1+\exp(-A_{I(i),I(i)})\cos(2a_{I(i)}))\\
C_{2i,2i}\;&=\;q_i(1-\exp(-A_{I(i),I(i)})\cos(2a_{I(i)}))\\
C_{2i,2i-1}\;=\;C_{2i-1,2i}\;&=\;-q_i\exp(-A_{I(i),I(i)})\sin(2a_{I(i)})\\
C_{2i-1,2j-1}\;&=\;q_{ij}([\exp(A_{I(i),I(j)})\!-\!1]\cos(a_{I(i)}\!-\!a_{I(j)})-
[\exp(-A_{I(i),I(j)})\!-\!1]\cos(a_{I(i)}\!+\!a_{I(j)}))\\
C_{2i,2j}\;&=\;q_{ij}([\exp(A_{I(i),I(j)})\!-\!1]\cos(a_{I(i)}\!-\!a_{I(j)})+
[\exp(-A_{I(i),I(j)})\!-\!1]\cos(a_{I(i)}\!+\!a_{I(j)}))\\
C_{2i,2j-1}\;&=\;q_{ij}(-[\exp(A_{I(i),I(j)})\!-\!1]\sin(a_{I(i)}
\!-\!a_{I(j)})+[\exp(-A_{I(i),I(j)})\!-\!1]\sin(a_{I(i)}\!+\!a_{I(j)}))\\
C_{2i-1,2j}\;&=\;q_{ij}([\exp(A_{I(i),I(j)})\!-\!1]\sin(a_{I(i)}
\!-\!a_{I(j)})+[\exp(-A_{I(i),I(j)})\!-\!1]\sin(a_{I(i)}\!+\!a_{I(j)})),
\end{split}
\]
where $q_i=(1-\exp(-A_{I(i),I(i)}))/2$, and $q_{ij}=\exp(-\tfrac{1}{2}(A_{I(i),I(i)}+A_{I(j),I(j)}))/2$.

% 2sin(A)sin(B) = cos(A-B)-cos(A+B)
% 2cos(A)cos(B) = cos(A-B)+cos(A+B)
% 2sin(A)cos(B) = sin(A-B)+cos(A+B)



\end{document}