
% This LaTeX was auto-generated from an M-file by MATLAB.
% To make changes, update the M-file and republish this document.



    
    
\begin{lstlisting}
function [L, dLdm, dLds, S2] = loss(cost, m, s)
\end{lstlisting}


\subsection*{Brief Description and Interface} 

\begin{par}
\textbf{Summary:} Cart-pole loss function. The loss is 1 - exp(-0.5*a*d\^{}2), where "a" is a (positive) constant and "d\^{}2" is the squared (Euclidean) distance between the tip of the pendulum and the upright position.
\end{par} \vspace{1em}
\begin{par}
If the exploration parameter "b" is not present (or if it is zero) then the expected loss is computed, averaging over the Gaussian distribution of the state, with mean "mu" and covariance matrix "Sigma". If it is present, then sum of the average loss and "b" times the std. deviation of the loss is returned. Negative values of "b" are used to encourage exploration and positive values avoid regions of uncertainty in the policy. Derivatives of these quantities are computed when desired. See also loss.pdf.
\end{par} \vspace{1em}
\begin{par}
inputs: m       mean of state distribution s       covariance matrix for the state distribution cost    cost structure   cost.p       length of the pendulum   cost.width   array of widths of the cost (summed together)   cost.expl    exploration parameter   cost.angle   array of angle indices   cost.target  target state
\end{par} \vspace{1em}
\begin{par}
outputs: L     expected cost dLdm  derivative of expected cost wrt. state mean vector dLds  derivative of expected cost wrt. state covariance matrix
\end{par} \vspace{1em}
\begin{par}
Copyright (C) 2008-2012 by and Marc Deisenroth Carl Edward Rasmussen, 2012-06-26. Edited by Joe Hall 2012-10-02.
\end{par} \vspace{1em}


\subsection*{Code} 


\begin{lstlisting}
cw = cost.width;
if ~isempty(cost.expl), b = cost.expl; else b = 0; end

% 1. Some precomputations
D0 = size(s,2); D = D0;                                  % state dimension
D1 = D0 + 2*length(cost.angle);           % state dimension (with sin/cos)

M = zeros(D1,1); M(1:D0) = m; S = zeros(D1); S(1:D0,1:D0) = s;
Mdm = [eye(D0); zeros(D1-D0,D0)]; Sdm = zeros(D1*D1,D0);
Mds = zeros(D1,D0*D0); Sds = kron(Mdm,Mdm);

% 2. Define static penalty as distance from target setpoint
ell = cost.p;
Q = zeros(D1); Q([1 D+1],[1 D+1]) = [1 ell]'*[1 ell]; Q(D+2,D+2) = ell^2;

% 3. Trigonometric augmentation
if D1-D0 > 0
  target = [cost.target(:); gTrig(cost.target(:), 0*s, cost.angle)];

  i = 1:D0; k = D0+1:D1;
  [M(k) S(k,k) C mdm sdm Cdm mds sds Cds] = gTrig(M(i),S(i,i),cost.angle);

  X = reshape(1:D1*D1,[D1 D1]); XT = X';              % vectorised indices
  I=0*X; I(i,i)=1; ii=X(I==1)'; I=0*X; I(k,k)=1; kk=X(I==1)';
  I=0*X; I(i,k)=1; ik=X(I==1)'; ki=XT(I==1)';

  Mdm(k,:)  = mdm*Mdm(i,:) + mds*Sdm(ii,:);                    % chainrule
  Mds(k,:)  = mdm*Mds(i,:) + mds*Sds(ii,:);
  Sdm(kk,:) = sdm*Mdm(i,:) + sds*Sdm(ii,:);
  Sds(kk,:) = sdm*Mds(i,:) + sds*Sds(ii,:);
  dCdm      = Cdm*Mdm(i,:) + Cds*Sdm(ii,:);
  dCds      = Cdm*Mds(i,:) + Cds*Sds(ii,:);

  S(i,k) = S(i,i)*C; S(k,i) = S(i,k)';                      % off-diagonal
  SS = kron(eye(length(k)),S(i,i)); CC = kron(C',eye(length(i)));
  Sdm(ik,:) = SS*dCdm + CC*Sdm(ii,:); Sdm(ki,:) = Sdm(ik,:);
  Sds(ik,:) = SS*dCds + CC*Sds(ii,:); Sds(ki,:) = Sds(ik,:);
end

% 4. Calculate loss!
L = 0; dLdm = zeros(1,D0); dLds = zeros(1,D0*D0); S2 = 0;
for i = 1:length(cw)                    % scale mixture of immediate costs
    cost.z = target; cost.W = Q/cw(i)^2;
  [r rdM rdS s2 s2dM s2dS] = lossSat(cost, M, S);

  L = L + r; S2 = S2 + s2;
  dLdm = dLdm + rdM(:)'*Mdm + rdS(:)'*Sdm;
  dLds = dLds + rdM(:)'*Mds + rdS(:)'*Sds;

  if (b~=0 || ~isempty(b)) && abs(s2)>1e-12
    L = L + b*sqrt(s2);
    dLdm = dLdm + b/sqrt(s2) * ( s2dM(:)'*Mdm + s2dS(:)'*Sdm )/2;
    dLds = dLds + b/sqrt(s2) * ( s2dM(:)'*Mds + s2dS(:)'*Sds )/2;
  end
end

% normalize
n = length(cw); L = L/n; dLdm = dLdm/n; dLds = dLds/n; S2 = S2/n;
\end{lstlisting}
