
% This LaTeX was auto-generated from an M-file by MATLAB.
% To make changes, update the M-file and republish this document.



    
    
      \subsection{gp2.m}

\begin{par}
\textbf{Summary:} Compute joint predictions and derivatives for multiple GPs with uncertain inputs. Does not consider the uncertainty about the underlying function (in prediction), hence, only the GP mean function is considered. Therefore, this representation is equivalent to a regularized RBF network. If gpmodel.nigp exists, individial noise contributions are added.
\end{par} \vspace{1em}
\begin{verbatim}function [M, S, V] = gp2(gpmodel, m, s)\end{verbatim}
\begin{par}
\textbf{Input arguments:}
\end{par} \vspace{1em}
\begin{verbatim}gpmodel    GP model struct
hyp(i)     struct array of GP hyper-parameters                   [ 1  x  E ]
    .l     log lengthscales                                      [ D  x  1 ]
    .s     log signal standard deviation                         [ 1  x  1 ]
    .n     log noise standard deviation                          [ 1  x  1 ]
    .m     linear weights for the GP mean                        [ D  x  1 ]
    .b     biases for the GP mean                                [ 1  x  1 ]
  inputs   training inputs                                       [ n  x  D ]
  targets  training targets                                      [ n  x  E ]
m          mean of the test distribution                         [ D  x  1 ]
s          covariance matrix of the test distribution            [ D  x  D ]\end{verbatim}
\begin{par}
\textbf{Output arguments:}
\end{par} \vspace{1em}
\begin{verbatim}M          mean of pred. distribution                            [ E  x  1 ]
S          covariance of the pred. distribution                  [ E  x  E ]
V          inv(s) times covariance between input and output      [ D  x  E ]\end{verbatim}
\begin{par}
Copyright (C) 2008-2014 by Marc Deisenroth, Andrew McHutchon, Joe Hall, and Carl Edward Rasmussen.
\end{par} \vspace{1em}
\begin{par}
Last modified: 2043-03-03
\end{par} \vspace{1em}


\subsection*{High-Level Steps} 

\begin{enumerate}
\setlength{\itemsep}{-1ex}
   \item Compute predicted mean and inv(s) times input-output covariance
   \item Compute predictive covariance matrix, non-central moments
   \item Centralize moments
\end{enumerate}

\begin{lstlisting}
function [M, S, V] = gp2(gpmodel, x, m, s)
\end{lstlisting}


\subsection*{Code} 


\begin{lstlisting}
[n, D, pE] = size(x); E = size(gpmodel.beta,2);
h = gpmodel.hyp; beta = gpmodel.beta;
if ~isfield(h,'m'); [h.m] = deal(zeros(D,1)); end
if ~isfield(h,'b'); [h.b] = deal(0); end

M = zeros(E,1); V = zeros(D,E); S = zeros(E);
k = zeros(n,E); a = zeros(D,E); M1 = zeros(E,1);

inp = bsxfun(@minus,x,m');                % x - m, either n-by-D or n-by-D-by-E


% 2) Compute predicted mean and inv(s) times input-output covariance
for i=1:E     % compute predicted mean and inv(s) times input-output covariance
  il = diag(exp(-h(i).l));          % Lambda^-1/2
  in = inp(:,:,min(i,pE))*il;       % (X - m)*Lambda^-1/2
  B = il*s*il+eye(D);               % Lambda^-1/2 * S * *Lambda^-1/2 + I

  t = in/B;                         % in.*t = (X-m) (S+L)^-1 (X-m)
  l = exp(-sum(in.*t,2)/2); lb = l.*beta(:,i);
  tL = t*il;
  c = exp(2*h(i).s)/sqrt(det(B));   % = sf2/sqrt(det(S*iL + I))

  M1(i) = sum(lb)*c; M(i) = M1(i) + h(i).m'*m + h(i).b;        % predicted mean
  V(:,i) = tL'*lb*c + h(i).m;            % inv(s) times input-output covariance
  k(:,i) = 2*h(i).s-sum(in.*in,2)/2;
  liBl = il*(B\il); xm = x(:,:,min(i,pE))'*lb*c;
  a(:,i) = diag(exp(2*h(i).l))*liBl*m*M1(i) + s*liBl*xm;
end

% 3) Compute predictive covariance, non-central moments
iL = exp(-2*[h.l]);inpiL = bsxfun(@times,inp,permute(iL,[3,1,2])); % N-by-D-by-E
for i=1:E                  % compute predictive covariance, non-central moments
  for j=1:i
    R = s*diag(iL(:,i)+iL(:,j))+eye(D); t = 1/sqrt(det(R));
    L = exp(bsxfun(@plus,k(:,i),k(:,j)')+maha(inpiL(:,:,i),-inpiL(:,:,j),R\s/2));
    S(i,j) = beta(:,i)'*L*beta(:,j)*t;                  % variance of the mean
    S(i,j) = S(i,j) + h(i).m'*(a(:,j) - m*M1(j)) + h(j).m'*(a(:,i) - m*M1(i));
    S(j,i) = S(i,j);
  end

  S(i,i) = S(i,i) + 1e-6;          % add small jitter for numerical reasons

end

% 4) Centralize moments
S = S - M1*M1' + [h.m]'*s*[h.m];                           % centralize moments
\end{lstlisting}
