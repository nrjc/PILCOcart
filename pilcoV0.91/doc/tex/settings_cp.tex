
% This LaTeX was auto-generated from an M-file by MATLAB.
% To make changes, update the M-file and republish this document.



    
    
      \subsection{settings\_cp.m}

\begin{par}
\textbf{Summary:} Script set up the cart-pole scenario
\end{par} \vspace{1em}
\begin{par}
Copyright (C) 2008-2014 by Marc Deisenroth, Andrew McHutchon, Joe Hall, and Carl Edward Rasmussen.
\end{par} \vspace{1em}
\begin{par}
Last modified: 2014-02-03
\end{par} \vspace{1em}


\subsection*{High-Level Steps} 

\begin{enumerate}
\setlength{\itemsep}{-1ex}
   \item Define state and important indices
   \item Set up scenario
   \item Set up the plant structure
   \item Set up the policy structure
   \item Set up the cost structure
   \item Set up the GP dynamics model structure
   \item Parameters for policy optimization
   \item Plotting verbosity
   \item Some array initializations
\end{enumerate}


\subsection*{Code} 


\begin{lstlisting}
varNames = {'x','dx','dangle','angle'};
rng(1); format short; format compact;
try
  rd = '../../';
  addpath([rd 'base'],[rd 'util'],[rd 'gp'],[rd 'control'],[rd 'loss']);
catch
end

% 1. Define state and important indices

% 1a. Full state representation (including all augmentations)
%
%  1  x          cart position
%  2  v          cart velocity
%  3  dtheta     angular velocity
%  4  theta      angle of the pendulum
%  5  sin(theta) complex representation ...
%  6  cos(theta) of theta
%  7  u          force applied to cart
%

% 1b. Important indices
% odei  indicies for the ode solver
% augi  indicies for variables augmented to the ode variables
% dyno  indicies for the output from the dynamics model and indicies to loss
% angi  indicies for variables treated as angles (using sin/cos representation)
% dyni  indicies for inputs to the dynamics model
% poli  indicies for the inputs to the policy

odei = [1 2 3 4];            % varibles for the ode solver
augi = [];                   % variables to be augmented
dyno = [1 2 3 4];            % variables to be predicted (and known to loss)
angi = 4;                    % angle variables
dyni = [1 2 3 4 5 6];        % variables that serve as inputs to the dynamics GP
poli = [1 2 3 5 6];          % variables that serve as inputs to the policy

% 2. Set up the scenario
dt = 0.10;                                             % [s] sampling time
T = 5.0;                             % [s] initial prediction horizon time
H = ceil(T/dt);                  % prediction steps (optimization horizon)
mu0 = [0 0 0 0]';                                     % initial state mean
S0 = diag([0.1 0.1 0.1 0.1].^2);                  % initial state variance
N = 15;                                  % number controller optimizations
J = 1;                                % initial J trajectories of length H
K = 1;                    % number of initial states for which we optimize
nc = 100;                           % number of controller basis functions
So = [0.01 0.01 pi/180 pi/180].^2; % measurement noise levels, 1cm, 1 degree


% 3. Plant structure
plant.ode = @dynamics_cp;                             % dynamics ode function
plant.noise = diag(So);                               % measurement noise
plant.dt = dt;
plant.ctrltype = @(t,f,f0)zoh(t,f,f0);    % ctrl implemented as zero order hold
plant.odei = odei;
plant.augi = augi;
plant.angi = angi;
plant.poli = poli;
plant.dyno = dyno;
plant.dyni = dyni;
plant.prop = @propagated;

% 4. Policy structure
policy.fcn = @(policy,m,s)conCat(@conGaussd,@gSat,policy,m,s);% controller
                                                              % representation
policy.maxU = 10;                                             % max. amplitude of
                                                              % control
mm = trigaug(mu0, zeros(length(mu0)), plant.angi);            % represent angles
                                                              % in complex plane
policy.p.cen = gaussian(mm(poli), eye(length(poli)), nc)';    % init. location of
                                                              % basis functions
policy.p.w = 0.1*randn(nc, length(policy.maxU));              % init basis fct
                                                              % weights
policy.p.ll = log([1 1 1 0.7 0.7])';                          % init length-scales

% %GP policy would need this:
% policy.p.inputs = gaussian(mm(poli), ss(poli,poli), nc)'; % init. location of
%                                                           % basis functions
% policy.p.targets = 0.1*randn(nc, length(policy.maxU));    % init. policy targets
%                                                           % (close to zero)
% policy.p.hyp = log([1 1 1 0.7 0.7 1 0.01])';              % initialize policy
%                                                           % hyper-parameters


% 5. Set up the cost structure
cost.fcn = @loss_cp;                       % cost function
cost.gamma = 1;                            % discount factor
cost.p = 0.5;                              % length of pendulum
cost.width = 0.25;                         % cost function width
cost.expl =  0.0;                          % exploration parameter (UCB)
cost.angle = plant.angi;                   % index of angle (for cost function)
cost.target = [0 0 0 pi]';                 % target state


% 6. Dynamics model structure
dynmodel.fcn = @gpBase;                % function for GP predictions
dynmodel.train = @train;               % function to train dynamics model
dynmodel.induce = zeros(300,0,1);      % shared inducing inputs (sparse GP)
trainOpt = [300 500];                  % max. no. of line searches
                                       % when training the GP
                                       % trainOpt(1): full GP
                                       % trainOpt(2): sparse GP (FITC)
dynmodel.approxS = 0;                  % approx. output covariance matrix ?
% 6.1: GP prior mean function
dynmodel.trainMean = 0;                % keep the GP prior mean fct fixed
                                       % during training
% only relevant if mean fct is fixed:
[dynmodel.hyp(1:length(dyno)).m] ...
  = deal(zeros(length(dyni)+length(policy.maxU), 1));
[dynmodel.hyp.b] = deal(0);
for i=1:length(dyno); dynmodel.hyp(i).m(i) = 1; end            % identity mean

% 7. Parameters for policy optimization
opt.length = 150;                        % max. number of line searches
opt.MFEPLS = 30;                         % max. number of function evaluations
                                         % per line search
opt.verbosity = 1;                       % verbosity: specifies how much
                                         % information is displayed during
                                         % policy learning. Options: 0-3
% 8. Plotting verbosity
plotting.verbosity = 1;            % 0: no plots
                                   % 1: some plots
                                   % 2: all plots

% 9. Some initializations
x = []; y = [];
fantasy.mean = cell(1,N); fantasy.std = cell(1,N);
realCost = cell(1,N); M = cell(N,1); Sigma = cell(N,1);
\end{lstlisting}
