
% This LaTeX was auto-generated from an M-file by MATLAB.
% To make changes, update the M-file and republish this document.



    
    
\begin{lstlisting}
function [L dLdm dLds S dSdm dSds C dCdm dCds dLdb] = lossHinge(cost, m, s)
% Function to compute the moments and derivatives of the loss of a Gaussian
% distributed point under a double hinge loss function. The loss function
% has slope -/+a and corners b1 and b2. The function also calculates derivatives
% of the loss w.r.t. the state distribution.
%
% Graph:
%          \                   /
%           \                 /
%            \               /
%          ,  \_____________/
%          0  b1           b2
%
% To use a single hinge b1 or b2 can be set to -Inf or +Inf respectively.
%
% Note, this function is only analytic for 1D inputs. To apply this loss
% function to multiple variables, use the lossAdd function.
%
% Inputs
%   cost
%       .fcn      @lossHinge - called to get here
%       .a        slope of loss function
%       .b        corner points of loss function, 1-by-2
%   m             input mean, D-by-1
%   S             input covariance matrix,    D-by-D
%
% Andrew McHutchon, Nov 2011

D = length(m);
if D > 1;
    error(['lossHinge only defined for 1D inputs, use lossAdd to '...
                                     'concatenate multiple 1D loss functions']);
end

a = cost.a;
b = cost.b(:)' - m(:)'; I = ~isinf(b); % centralise
eb = exp(-b.^2/2/s); erfb = erf(b/sqrt(2*s));
c = sqrt(s/pi/2);

% Expected Loss
% int_{-inf}^{b1-m} -a*(x-b1+m)*N(0,S)  +  int_{b2-m}^inf a*(x-b2+m)*N(0,S)
L = a*(b/2.*erfb + c*eb + b.*[1,-1]/2);
L = sum(L(I));

if nargout > 1
    % Derivative w.r.t. m
    dLdb = a/2*(erfb + [1,-1]);
    dLdm = sum(dLdb(I)*-1);

    % Derivative w.r.t. S
    dc = 1/(2*sqrt(2*pi*s));
    dLds = a*sum(eb)*dc;
end

% Variance of Loss
if nargout > 3
    S = a^2*((b.^2+s).*(1+[1,-1].*erfb)/2 + [1,-1].*b*c.*eb);
    S = sum(S(I)) - L^2;

    erfbdm = -sqrt(2/pi/s)*eb; erfbds = -b.*eb/sqrt(2*pi*s^3);
    dSdm = a^2*(-b.*(1+[1,-1].*erfb) + (b.^2+s).*[1,-1].*erfbdm/2 + ...
                                                    [1,-1]*c.*eb.*(-1+b.^2/s));
    dSdm = sum(dSdm(I)) - 2*L*dLdm;
    dSds = a^2/2*((1+[1,-1].*erfb) + (b.^2+s).*[1,-1].*erfbds + ...
                                   [2,-2].*b*dc.*eb + [1,-1].*b.^3/s^2*c.*eb);
    dSds = sum(dSds(I)) - 2*L*dLds;
end

% inv(s)*IO covariance
if nargout > 6
   C = a*([-1 1] - erfb)/2;
   C = sum(C);

   dCdm = -a/2*sum(erfbdm);
   dCds = -a/2*sum(erfbds(I));
end
\end{lstlisting}
