
% This LaTeX was auto-generated from an M-file by MATLAB.
% To make changes, update the M-file and republish this document.



    
    
      \subsection{simulate.m}

\begin{par}
\textbf{Summary:} Simulate dynamics using a given control scheme.
\end{par} \vspace{1em}

\begin{verbatim}  function next = simulate(x0, f, plant)\end{verbatim}
    \begin{par}
Compute the next discrete time state from the current state and action by solving the continuous time plant dynamics ode. The type of implementation for the controller is specified in plant.ctrltype, one of zoh (Zero Order Hold), foh (First Order Hold) or expdecay (exponential decay), defined below, which turn the discrete time control action into a continuous time control signal. If no plant.delay exists, the ode-solver (ode45) is called directly with the plant.ctrltype, otherwise a time delay is introduced, and the control actions from the previous time interval (saved in persistent variable U) is mixed together with the new action, ensuring a smooth transition (the transition takes about one 100th of the time interval) to avoid creating difficulties for the ode solver at the transition.
\end{par} \vspace{1em}
\begin{par}
\textbf{Input arguments:}
\end{par} \vspace{1em}
\begin{par}
x0           initial state f            vector of control actions plant        plant structure; fields used by this function are   .dt         time discretization   .ode        ordinary differential equation describing the system dynamics   .ctrltype   function defining control implementation, possible values      @zoh       zero-order-hold control (ZOH)      @foh       first-order-hold control (FOH) with rise time      @expdecay  lagged control with time constant   .delay      optional, continuous-time delay, in range [0 dt)
\end{par} \vspace{1em}
\begin{par}
\textbf{Output arguments:}
\end{par} \vspace{1em}

\begin{verbatim}next    successor state (with additional control states if required)\end{verbatim}
    \begin{par}
Copyright (C) 2008-2014 by Marc Deisenroth, Andrew McHutchon, Joe Hall, and Carl Edward Rasmussen.
\end{par} \vspace{1em}
\begin{par}
Last modification: 2013-12-16
\end{par} \vspace{1em}


\subsection*{High-Level Steps} 

\begin{par}
For each time step \# Set up the control function \# Simulate the dynamics (by calling ODE45)
\end{par} \vspace{1em}

\begin{lstlisting}
function next = simulate(x0, f, plant)
\end{lstlisting}


\subsection*{Code} 


\begin{lstlisting}
OPTIONS = odeset('RelTol', 1e-12, 'AbsTol', 1e-12);         % accuracy of ode45
persistent U;                                % previous control action function
dt = plant.dt; eval(['ctrltype = ', func2str(plant.ctrltype), ';']);   % scope!

% 1. Set up control function ------------------------------------------------
if isempty(U), U = @(t)ctrltype(t+dt, 0*f, 0*f); end; U0 = U(0);

% 1a (optional) handle delay
if isfield(plant, 'delay')
  d = plant.delay; s = -100/dt;   % set steepness of transition in mix function
  mix = @(t, a, b)(a(t-d)*exp(s*t)+b(t-d)*exp(s*d))/(exp(s*t)+exp(s*d));
  u = @(t)mix(t, U, @(t)ctrltype(t, f, U0));             % mix actions smoothly
else
  u = @(t)ctrltype(t, f, U0);
end

% 2. Simulate dynamics ------------------------------------------------------
[T, y] = ode45(plant.ode, [0 dt/2 dt], x0, OPTIONS, u);          % solve the ode
next = y(3,:)';                                       % extract the desired state
U = @(t)ctrltype(t+dt, f, U0);               % save the control action function
\end{lstlisting}

\begin{lstlisting}
function u = zoh(t, f, f0)                            % zero order hold control
u = f;

function u = foh(t, f, f0, risetime) % first order hold with specified risetime
if t < risetime, u = f0 + t*(f-f0)/risetime; else u = f; end

function u = expdecay(t, f, f0, tau) % exponential decay with time constant tau
u = f + exp(-t/tau)*(f0-f);
\end{lstlisting}
