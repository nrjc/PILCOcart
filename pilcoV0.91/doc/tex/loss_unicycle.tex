
% This LaTeX was auto-generated from an M-file by MATLAB.
% To make changes, update the M-file and republish this document.



    
    
\begin{lstlisting}
function [L, dLdm, dLds, S2] = loss(cost, m, s)
% Robotic unicycle loss function. The loss is 1 - exp(-0.5*a*d^2), where
% "a"is a (positive) constant and "d^2" is the squared difference between
% the current z-position of the top of the unicycle and the upright
% position.
%
% Compute the expected loss, averaged over a Gaussian state distibution,
% plus cost.expl times the standard deviation of the loss (averaged wrt
% the same Gaussian), where the exploration paramater cost.expl defaults
% to zero.
%
% Negative values of the exploration parameter are used to encourage
% exploration and positive values avoid regions of uncertainty in the
% policy. Derivatives are computed when desired.
%
% inputs:
% m       mean of state distribution
% s       covariance matrix for the state distribution
% cost    cost structure
%   cost.p       parameters: [radius of wheel, length of rod]
%   cost.width   array of widths of the cost (summed together)
%   cost.expl    exploration parameter
%
% outputs:
% L     expected cost
% dLdm  derivative of expected cost wrt. state mean vector
% dLds  derivative of expected cost wrt. state covariance matrix
%
% Copyright (C) 2009-2012 Carl Edward Rasmussen, Marc Deisenroth
%            and Philipp Hennig, 2012-03-28. Edited by Joe Hall 2012-10-02

I6 = 8;  I9 = 10;                           % coordinates of theta and psi
Ixc = 6; Iyc = 7;                               % coordinates of xc and yc

cw = cost.width; rw = cost.p(1); r = cost.p(2);
if isfield(cost,'expl'), b = cost.expl; else b = 0; end

% 1. Some precomputations
D = size(s,2);                                           % state dimension
D0 = D + 2;             % state dimension (augmented with I6-I9 and I6+I9)
D1 = D0 + 8;                              % state dimension (with sin/cos)
L = 0; dLdm = zeros(1,D); dLds = zeros(1,D*D); S2 = 0;

P = [eye(D); zeros(2,D)]; P(D+1:end,I6) = [1;-1]; P(D+1:end,I9) = [1;1];

M = zeros(D1,1); M(1:D0) = P*m; S = zeros(D1); S(1:D0,1:D0) = P*s*P';
Mdm = [P; zeros(D1-D0,D)]; Sdm = zeros(D1*D1,D);
Mds = zeros(D1,D*D); Sds = kron(Mdm,Mdm);

% 2. Define static penalty as distance from target setpoint
Q = zeros(D+10);
C1 = [rw r/2 r/2];
Q([D+4 D+6 D+8],[D+4 D+6 D+8]) = 8*(C1'*C1);                          % dz
C2 = [1 -r];
Q([Ixc D+9],[Ixc D+9]) = 0.5*(C2'*C2);                                % dx
C3 = [1 -(r+rw)];
Q([Iyc D+3],[Iyc D+3]) = 0.5*(C3'*C3);                                % dy
Q(9,9) = (1/(4*pi))^2;                                    % yaw angle loss

target = zeros(D1,1); target([D+4 D+6 D+8 D+10]) = 1;    % target setpoint

% 3. Trigonometric augmentation
i = 1:D0; k = D0+1:D1;
[M(k) S(k,k) C mdm sdm Cdm mds sds Cds] = ...
                                       gTrig(M(i),S(i,i),[I6 D+1 D+2 I9]);
[S Mdm Mds Sdm Sds] = ...
               fillIn(S,C,mdm,sdm,Cdm,mds,sds,Cds,Mdm,Sdm,Mds,Sds,i,k,D1);

% 4. Calculate loss!
for i = 1:length(cw)                    % scale mixture of immediate costs
    cost.z = target; cost.W = Q/cw(i)^2;
  [r rdM rdS s2 s2dM s2dS] = lossSat(cost, M, S);

  L = L + r; S2 = S2 + s2;
  dLdm = dLdm + rdM(:)'*Mdm + rdS(:)'*Sdm;
  dLds = dLds + rdM(:)'*Mds + rdS(:)'*Sds;

  if (b~=0 || ~isempty(b)) && abs(s2)>1e-12
    L = L + b*sqrt(s2);
    dLdm = dLdm + b/sqrt(s2) * ( s2dM(:)'*Mdm + s2dS(:)'*Sdm )/2;
    dLds = dLds + b/sqrt(s2) * ( s2dM(:)'*Mds + s2dS(:)'*Sds )/2;
  end
end

% normalize
n = length(cw); L = L/n; dLdm = dLdm/n; dLds = dLds/n; S2 = S2/n;

% Fill in covariance matrix...and derivatives ----------------------------
function [S Mdm Mds Sdm Sds] = ...
                 fillIn(S,C,mdm,sdm,Cdm,mds,sds,Cds,Mdm,Sdm,Mds,Sds,i,k,D)
X = reshape(1:D*D,[D D]); XT = X';                    % vectorised indices
I=0*X; I(i,i)=1; ii=X(I==1)'; I=0*X; I(k,k)=1; kk=X(I==1)';
I=0*X; I(i,k)=1; ik=X(I==1)'; ki=XT(I==1)';

Mdm(k,:)  = mdm*Mdm(i,:) + mds*Sdm(ii,:);                      % chainrule
Mds(k,:)  = mdm*Mds(i,:) + mds*Sds(ii,:);
Sdm(kk,:) = sdm*Mdm(i,:) + sds*Sdm(ii,:);
Sds(kk,:) = sdm*Mds(i,:) + sds*Sds(ii,:);
dCdm      = Cdm*Mdm(i,:) + Cds*Sdm(ii,:);
dCds      = Cdm*Mds(i,:) + Cds*Sds(ii,:);

S(i,k) = S(i,i)*C; S(k,i) = S(i,k)';                        % off-diagonal
SS = kron(eye(length(k)),S(i,i)); CC = kron(C',eye(length(i)));
Sdm(ik,:) = SS*dCdm + CC*Sdm(ii,:); Sdm(ki,:) = Sdm(ik,:);
Sds(ik,:) = SS*dCds + CC*Sds(ii,:); Sds(ki,:) = Sds(ik,:);
\end{lstlisting}
