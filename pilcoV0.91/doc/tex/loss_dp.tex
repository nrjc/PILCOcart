
% This LaTeX was auto-generated from an M-file by MATLAB.
% To make changes, update the M-file and republish this document.



    
    
      \subsection{loss\_dp.m}

\begin{par}
\textbf{Summary:} Double-Pendulum loss function; the loss is $1-\exp(-0.5*d^2*a)$,  where $a>0$ and  $d^2$ is the squared difference between the actual and desired position of the tip of the outer pendulum. The mean and the variance of the loss are computed by averaging over the Gaussian distribution of the state $p(x) = \mathcal N(m,s)$ with mean $m$ and covariance matrix $s$. Derivatives of these quantities are computed when desired.
\end{par} \vspace{1em}
\begin{verbatim}function [L, dLdm, dLds, S2] = loss\_dp(cost, m, s)\end{verbatim}
\begin{par}
\textbf{Input arguments:}
\end{par} \vspace{1em}
\begin{verbatim}cost            cost structure
  .p            lengths of the 2 pendulums                      [2 x  1 ]
  .width        array of widths of the cost (summed together)
  .expl         (optional) exploration parameter
  .angle        (optional) array of angle indices
  .target       target state                                    [D x  1 ]
m               mean of state distribution                      [D x  1 ]
s               covariance matrix for the state distribution    [D x  D ]\end{verbatim}
\begin{par}
\textbf{Output arguments:}
\end{par} \vspace{1em}
\begin{verbatim}L     expected cost                                             [1 x  1 ]
dLdm  derivative of expected cost wrt. state mean vector        [1 x  D ]
dLds  derivative of expected cost wrt. state covariance matrix  [1 x D^2]
S2    variance of cost                                          [1 x  1 ]\end{verbatim}
\begin{par}
Copyright (C) 2008-2013 by Marc Deisenroth, Andrew McHutchon, Joe Hall, and Carl Edward Rasmussen.
\end{par} \vspace{1em}
\begin{par}
Last modified: 2013-03-08
\end{par} \vspace{1em}


\subsection*{High-Level Steps} 

\begin{enumerate}
\setlength{\itemsep}{-1ex}
   \item Precomputations
   \item Define static penalty as distance from target setpoint
   \item Trigonometric augmentation
   \item Calculate loss
\end{enumerate}

\begin{lstlisting}
function [L, dLdm, dLds, S2] = loss_dp(cost, m, s)
\end{lstlisting}


\subsection*{Code} 


\begin{lstlisting}
if isfield(cost,'width'); cw = cost.width; else cw = 1; end
if ~isfield(cost,'expl') || isempty(cost.expl); b = 0; else b =  cost.expl; end


% 1. Some precomputations
D0 = size(s,2); D = D0;                                  % state dimension
D1 = D0 + 2*length(cost.angle);           % state dimension (with sin/cos)

M = zeros(D1,1); M(1:D0) = m; S = zeros(D1); S(1:D0,1:D0) = s;
Mdm = [eye(D0); zeros(D1-D0,D0)]; Sdm = zeros(D1*D1,D0);
Mds = zeros(D1,D0*D0); Sds = kron(Mdm,Mdm);

% 2. Define static penalty as distance from target setpoint
ell1 = cost.p(1); ell2 = cost.p(2); C = [ell1 0 ell2 0; 0 ell1 0 ell2];
Q = zeros(D1); Q(D+1:D+4,D+1:D+4) = C'*C;

% 3. Trigonometric augmentation
if D1-D0 > 0
  target = [cost.target(:); gTrig(cost.target(:), 0*s, cost.angle)];

  i = 1:D0; k = D0+1:D1;
  [M(k) S(k,k) C mdm sdm Cdm mds sds Cds] = gTrig(M(i),S(i,i),cost.angle);
  [S Mdm Mds Sdm Sds] = ...
               fillIn(S,C,mdm,sdm,Cdm,mds,sds,Cds,Mdm,Sdm,Mds,Sds,i,k,D1);
end

% 4. Calculate loss
L = 0; dLdm = zeros(1,D0); dLds = zeros(1,D0*D0); S2 = 0;
for i = 1:length(cw)                    % scale mixture of immediate costs
    cost.z = target; cost.W = Q/cw(i)^2;
  [r rdM rdS s2 s2dM s2dS] = lossSat(cost, M, S);

  L = L + r; S2 = S2 + s2;
  dLdm = dLdm + rdM(:)'*Mdm + rdS(:)'*Sdm;
  dLds = dLds + rdM(:)'*Mds + rdS(:)'*Sds;

  if (b~=0 || ~isempty(b)) && abs(s2)>1e-12
    L = L + b*sqrt(s2);
    dLdm = dLdm + b/sqrt(s2) * ( s2dM(:)'*Mdm + s2dS(:)'*Sdm )/2;
    dLds = dLds + b/sqrt(s2) * ( s2dM(:)'*Mds + s2dS(:)'*Sds )/2;
  end
end

% normalize
n = length(cw); L = L/n; dLdm = dLdm/n; dLds = dLds/n; S2 = S2/n;

% Fill in covariance matrix...and derivatives ----------------------------
\end{lstlisting}

\begin{lstlisting}
function [S Mdm Mds Sdm Sds] = ...
                 fillIn(S,C,mdm,sdm,Cdm,mds,sds,Cds,Mdm,Sdm,Mds,Sds,i,k,D)
X = reshape(1:D*D,[D D]); XT = X';                    % vectorized indices
I=0*X; I(i,i)=1; ii=X(I==1)'; I=0*X; I(k,k)=1; kk=X(I==1)';
I=0*X; I(i,k)=1; ik=X(I==1)'; ki=XT(I==1)';

Mdm(k,:)  = mdm*Mdm(i,:) + mds*Sdm(ii,:);                      % chainrule
Mds(k,:)  = mdm*Mds(i,:) + mds*Sds(ii,:);
Sdm(kk,:) = sdm*Mdm(i,:) + sds*Sdm(ii,:);
Sds(kk,:) = sdm*Mds(i,:) + sds*Sds(ii,:);
dCdm      = Cdm*Mdm(i,:) + Cds*Sdm(ii,:);
dCds      = Cdm*Mds(i,:) + Cds*Sds(ii,:);

S(i,k) = S(i,i)*C; S(k,i) = S(i,k)';                        % off-diagonal
SS = kron(eye(length(k)),S(i,i)); CC = kron(C',eye(length(i)));
Sdm(ik,:) = SS*dCdm + CC*Sdm(ii,:); Sdm(ki,:) = Sdm(ik,:);
Sds(ik,:) = SS*dCds + CC*Sds(ii,:); Sds(ki,:) = Sds(ik,:);
\end{lstlisting}
