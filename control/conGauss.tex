\documentclass[a4paper, 11pt]{article}

\title{RBF Network Controller}
\date{\today}
\author{Marc Deisenroth}


\usepackage{algorithm}
\usepackage{algorithmic}


\usepackage{units}
\usepackage{url}
\usepackage{graphicx}
\usepackage{mathtools}
\usepackage{dsfont}
\usepackage{amsmath, amssymb}

\usepackage{geometry}







\newcommand{\R}{\mathds{R}}
\newcommand{\Z}{\mathds{Z}}
\renewcommand{\vec}{\boldsymbol}
\newcommand{\mat}{\boldsymbol}
\newcommand{\E}{\mathds{E}}
\newcommand{\var}{\mathrm{var}}
\newcommand{\cov}{\mathrm{cov}}
\newcommand{\diag}{\mathrm{diag}}
\newcommand{\tr}{\mathrm{tr}}
\newcommand{\cost}{c}
\newcommand{\T}{^\top}
\newcommand{\inv}{^{-1}}
\newcommand{\prob}{{p}}
\renewcommand{\d}{\mathrm d}
\newcommand{\gauss}[2]{\mathcal N(#1,#2)}
\newcommand{\gaussx}[3]{\mathcal{N}\big(#1\,|\,#2,#3\big)}
\renewcommand{\d}{\operatorname{d}\!}









\begin{document}
\maketitle


In the nonlinear case, we represent the preliminary policy $\tilde\pi$
by a radial basis function network with Gaussian basis functions. The
preliminary RBF policy is given by
%
\begin{equation}\label{eq:policy RBF}
  \tilde\pi(\vec x_*) =\sum\limits_{s=1}^N w_s \phi(\vec\mu_s,\vec x_*) =
  \vec w_\pi\T \vec\phi(\mat M_\pi,\vec x_*)\,,
\end{equation}
%
where $\vec x_*$ is a test input, $\phi(\vec\mu_s,\vec x_*) =
\exp(-(\vec \mu_s-\vec x_*)\T\mat\Lambda\inv(\vec\mu_s-\vec x_*)/2)$ is
an unnormalized Gaussian basis function centered at $\vec \mu_s$,
$\mat\Lambda$ is a matrix of squared length-scales,  and
$\vec w_\pi$ is a weight vector.

The set $\mat M_\pi=[\vec \mu_1,\dotsc,\vec \mu_N]$, $\vec
\mu_s\in\R^D$, $s=1,\dotsc,N$, are the locations of the centers of the
Gaussian basis functions, also called the \emph{support points}. The
RBF network in Eq.~(\ref{eq:policy RBF}) allows for flexible modeling,
which is useful if the structure of the underlying function (in our
case a good policy) is unknown.


% \paragraph{Remark: Interpretation as a deterministic GP}
% % RBF network as deterministic GP
%   The RBF network given in \eq~(\ref{eq:policy RBF}) is
%   functionally equivalent to the mean function. Thus, the RBF network
%   can be considered a ``deterministic GP'' with a fixed number of $N$
%   basis functions. Here, ``deterministic'' means that there is no
%   uncertainty about the underlying function, that is,
%   $\var_{\tilde\pi}[\tilde\pi(\vec x)]=0$. Note, however, that the RBF
%   network is a finite and degenerate model; the predicted variances
%   far away from the centers of the basis functions decline to zero.

\paragraph{Predictive Distribution.}
For a Gaussian distributed state $\vec x_*\sim\gauss{\vec
  m_*}{\mat\Sigma_*}$, the predictive mean of $\tilde\pi(\vec x_*)$ is
given as
%
\begin{align}
\E_{\vec x_*}[\tilde\pi(\vec x_*)] &= \vec w_\pi\T \E_{\vec
  x_*}[\vec\phi(\mat M_\pi, \vec x_*)]\\
&= \vec w_\pi\T\int \vec\phi(\mat M_\pi, \vec x_*)\prob(\vec
x_*)\d\vec x_* =\vec w_\pi\T\vec q\,,
\label{eq:pred mean prel pol}
\end{align}
where for $i=1,\dotsc,N$, and $a = 1,\dotsc, F$
\begin{align}
q_{a_i} &= |\mat\Sigma_*\mat\Lambda_a\inv + \mat
I|^{-\tfrac{1}{2}}\exp(-\tfrac{1}{2}(\vec\mu_*
-\vec\mu_i)\T(\mat\Sigma_* + \mat\Lambda_a)\inv(\vec\mu_*
-\vec\mu_i))\,,
\end{align}
where $\vec\mu_i$, $i=1\dotsc,N$ are the centers of the axis-aligned
Gaussian basis functions of the RBF network, and $\mat\Lambda_a$ is a
diagonal matrix of squared length-scales.

The predictive covariance matrix is computed according to
%
\begin{align*}
\var_{\vec x_*}[\tilde\pi(\vec x_*)]&\!=\!\E_{\vec x_*}[\tilde\pi(\vec
x_*)\tilde\pi(\vec x_*)\T]\! -\! \E_{\vec x_*}[\tilde\pi(\vec
x_*)]\E_{\vec x_*}[\tilde\pi(\vec
x_*)]\T,
\end{align*}
% 
where $\E_{\vec x_*}[\tilde\pi(\vec x_*)]$ is given in
Eq.~{eq:pred mean prel pol}). Hence, we focus on the term
$\E_{\vec x_*}[\tilde\pi(\vec x_*)\tilde\pi(\vec x_*)\T]$, which is given by
%
\begin{align*}
&\vec w_\pi\T\E_{\vec x_*}[\mat\phi(\mat M_\pi, \vec x_*)\mat\phi(\mat
M_\pi, \vec x_*)\T]\vec w_\pi=\vec w_\pi\T\mat Q\vec w_\pi\,,
\end{align*}
% 
where for $i,j = 1,\dotsc, N$, we compute
%
\begin{align*}
Q_{ij} &= \int\phi(\vec\mu_i, \vec x_*)\phi(\vec\mu_j, \vec
x_*)\prob(\vec x_*)\d\vec x_*\\
&=\phi(\vec\mu_i,\vec\mu_*)\phi(\vec\mu_j,\vec\mu_*)|\mat R|^{-\tfrac{1}{2}}\exp(\vec
z_{ij}\T\mat R\inv\mat\Sigma_*\vec z_{ij})\,,\\
\mat R &= \mat\Sigma_*(\mat\Lambda_a\inv + \mat\Lambda_b\inv) + \mat
I\,,\\
\vec z_{ij} &= \mat\Lambda_a\inv(\vec\mu_*-\vec\mu_i) +
\mat\Lambda_b\inv(\vec\mu_*-\vec\mu_j)\,.
\end{align*}
%
Combining this result with Eq.~(\ref{eq:pred mean prel pol}),
determines the predictive covariance matrix of the preliminary policy.




\end{document}


%%% Local Variables: 
%%% mode: latex
%%% TeX-master: t
%%% End: 
